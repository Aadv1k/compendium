\documentclass[12pt, letterpaper]{article}
\usepackage{amsmath}

\title{A Short \& Sweet Introduction to Graph Theory by a Math Dropout}
\author{}
\date{23/03/2024}

\newtheorem{theorem}{Theorem}

\begin{document}
\maketitle

\section*{Introduction}
A graph is an abstract construct where vertices connect with an indefinite number of vertices through edges. Formally, it can be defined as a set of vertices and a set of edges, where each edge is a pair of vertices.

\[G = (V, E) \]
\[V = \{1, 2, 3, 4\} \]
\[E = \{\{1, 2\}, \{3, 4\}\} \]

\begin{itemize}
    \item An edge is said to be \textbf{incident} to a vertex if it connects with it.
    \item \textbf{Degree} is defined as the number of edges incident to a particular vertex.    
\end{itemize}

\begin{theorem}
\label{thrm:sum-deg-equals-twice-edges}
The sum of the degrees of all vertices in graph $G$ is always equal to twice the number of edges.
\[
\sum_{v \in V} deg(v) = 2|E|
\]
\end{theorem}

\textbf{Intuitive Proof:} the theorem states that the sum of degrees of all the vertices is always twice the number of edges. We know that an edge always connects two vertices; hence, whenever a new edge is added to a graph, the degree of each of the two vertices it connects increases by one, contributing two to the total sum.

\begin{proof}
    This is also known as the \textbf{Handshake Lemma}
    
    Let $G = (V, E)$ where $V$ are the vertices, $E$ are the edges.

    \[
    \forall e \in E \quad \text{where} \quad e = (v_i, v_j) \in V 
    \]

    We know, every edge connects two vertices, so every edge contributes two to the total degree for $G$
    \[
    \frac{\sum_{v \in V} deg(v)}{2} = |E|
    \]
    \[
    \sum_{v \in V} deg(v) = 2|E|
    \]
\end{proof}

\end{document}
